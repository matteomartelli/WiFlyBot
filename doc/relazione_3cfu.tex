\documentclass[a4paper,11pt]{report}
\usepackage[T1]{fontenc}
\usepackage[utf8]{inputenc}
\usepackage{lmodern}
\usepackage{hyperref}

\title{TITOLO PROVVISORIO}
\author{Matteo Martelli}

\begin{document}

\maketitle
\tableofcontents

\begin{abstract}
\end{abstract}

\chapter{Introduzione}

\section{Scenari di Disaster Recovery}

\section{STEMNET}

\section{Oltre la simulazione}

\chapter{Piano di Processo}

\section{Motivazioni}

\section{Obbiettivi}
Gli obbiettivi sono..

\section{Risorse e strumenti impiegati}
\begin{itemize}
  \item Hardware
  \begin {itemize}
    \item Arduino Uno: piattaforma hardware programmabile tramite seriale USB. 
      Monta un microcontrollore ATmega328P, 14 pin I/O digitali e 6 pin I/O analogici.\\
      \url{http://arduino.cc/en/Main/ArduinoBoardUno} 
    \item Magician Chassis ROB-10825: piattaforma robot. Monta due motori con ruote da 65mm e una rotella posteriore.\\
        \url{https://www.sparkfun.com/products/10825}
    \item Motor Driver 1A Dual TB6612FNG: scheda per il controllo dei motori. Può controllare fino a due motori.\\
        \url{https://www.sparkfun.com/products/9457}
    \item Wireless Proto Shield: scheda per connettere facilmente dei moduli wireless all'Arduino board.\\
        \url{http://arduino.cc/en/Main/ArduinoWirelessProtoShield}
    \item WiFly Shield RN-XV: scheda che incorpora il modulo Wi-Fi RN171. Permette di connettersi ad una rete Wi-Fi 
        in modalità infrastructure o Ad-Hoc.\\
        \url{http://rovingnetworks.com/products/RN171XV}
    \item Due interfaccie Wi-Fi su sistema GNU Linux.  
  \end {itemize}
  \item Software e Linguaggi di Programmazione
  \begin {itemize}
    \item Wiring per Arduino: è una piattaforma di sviluppo open source composta da un linguaggio di programmazione derivato da C e C++ ed un ambiente di sviluppo integrato (Integrated Development Environment o IDE) con elementi per gestire l'hardware dell'Arduino Board.
    \item Python per la parte di applicazione GNU Linux.
  \end{itemize}
  
\end{itemize}

\section{Scenario di STEMNET}

\section{Scenario di Progetto}
Nel progetto si fa riferimento ad uno scenario semplificato rispetto a quello esposto in STEMNET.
Ci si vuole ridurre al caso di un singolo STEM-Node (SN) in modo da poterne studiare nello specifico le possibilità implementative.
L'idea è quella di posizionare lo SN tra altri due nodi Wi-Fi, i quali simulerebbero il comportamento di altri SN o delle isole.
Applicando 

\chapter{Progettazione}


\end{document}
